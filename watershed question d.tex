\documentclass[12pt, a4paper]{article}

% --- PAGE SETTINGS ---
\usepackage[a4paper, margin=1in]{geometry}

% --- FONT: Times New Roman ---
\usepackage{newtxtext,newtxmath}

% --- BASIC PACKAGES ---
\usepackage{tabularx}
\usepackage{amssymb}
\usepackage{ulem}
\usepackage{enumitem}
\usepackage{amsmath}
\usepackage{array}
\usepackage{xcolor}
\usepackage{colortbl}

% --- DOCUMENT START ---
\begin{document}

% --- HEADER SECTION ---
\begin{center}
    \large\textbf{TRIBHUVAN UNIVERSITY} \\
    \large\textbf{INSTITUTE OF ENGINEERING} \\
    \textbf{Examination Control Division} \\
    \textbf{2083 Baishakh}
\end{center}
\vspace{2mm}

% --- INFORMATION TABLE ---
\noindent
\begin{tabular}{|l|p{6cm}|l|p{4cm}|}
    \hline
    \textbf{Exam} & \multicolumn{3}{>{\centering\arraybackslash}p{10.5cm}|}{\cellcolor{black}\textcolor{white}{\textbf{Regular / Back}}} \\
    \hline
    \textbf{Level} & BE & \textbf{Full Marks} & 80 \\
    \hline
    \textbf{Programme} & BCE & \textbf{Pass Marks} & 32 \\
    \hline
    \textbf{Year / Part} & IV / I & \textbf{Time} & 3 hrs. \\
    \hline
\end{tabular}
\vspace{4mm}

% --- SUBJECT TITLE ---
\hrule
\vspace{2mm}
\begin{center}
    \textbf{Subject: - Soil Conservation and Watershed Management}
\end{center}
\vspace{-2mm}
\hrule
\vspace{5mm}

% --- INSTRUCTIONS ---
\begin{itemize}[leftmargin=*, label=\checkmark]
    \item Candidates are required to give their answers in their own words as far as practicable.
    \item Attempt \uline{All} questions.
    \item The figures in the margin indicate \uline{Full Marks}.
    \item Assume suitable data if necessary.
\end{itemize}
\vspace{5mm}
\hrule
\vspace{5mm}

% --- GROUP A ---
\section*{Group A: Theory and Concepts (40 Marks)}

\begin{enumerate}
    \item What factors influence watershed management? Describe the physiographic and climatic factors in detail. \hfill \textbf{[5]}

    \item Explain the different types of soil erosion caused by water, starting from raindrop erosion and progressing to gully erosion. \hfill \textbf{[5]}

    \item What are the primary factors that limit the capability of land for agricultural use? \hfill \textbf{[4]}

    \item Explain the concept of agroforestry and discuss its role in soil and water conservation. \hfill \textbf{[5]}

    \item What are the main engineering principles for controlling erosion on agricultural land? \hfill \textbf{[5]}

    \item Describe the main components of a chute spillway and explain the purpose of creating a hydraulic jump at its outlet. \hfill \textbf{[5]}

    \item With the aid of neat sketches, explain the purpose and construction of any \textbf{three} of the following bio-engineering measures:
    \begin{itemize}
        \item Fascines
        \item Palisades
        \item Live Fencing
        \item Rip-rap
    \end{itemize}
    \hfill \textbf{[6]}

    \item What is groundwater recharge? Describe two common methods used for artificially recharging groundwater. \hfill \textbf{[5]}
\end{enumerate}

\vspace{5mm}
\hrule
\vspace{5mm}

% --- GROUP B ---
\section*{Group B: Numerical Problems (40 Marks)}

\begin{enumerate}[resume]
    \item A field plot study provides the following seasonal data. If no soil conservation practice is used and the topographic factor (LS) is 1.15, determine the soil erodibility factor (K) for each season and then calculate the average K value for the plot.
    \begin{center}
        \begin{tabular}{|l|c|c|c|}
            \hline
            \textbf{Season} & \textbf{Soil Loss A (t/ha)} & \textbf{R-Factor (R)} & \textbf{C-Factor (C)} \\
            \hline
            Winter & 10.0 & 35 & 0.35 \\
            \hline
            Spring & 7.55 & 350 & 0.50 \\
            \hline
            Summer & 1.5 & 750 & 0.40 \\
            \hline
            Fall & 0.15 & 110 & 0.20 \\
            \hline
        \end{tabular}
    \end{center}
    \hfill \textbf{[6]}

    \item Determine the peak runoff rate for a 25-year return period to design a gully control structure for a watershed of 10 km$^2$. The length of the watershed channel is 10,000 m and the catchment slope is 0.5\%. The average runoff coefficient is 0.45. Use the rainfall data provided below.
    \begin{center}
        \begin{tabular}{|l|c|c|c|c|c|c|c|}
            \hline
            \textbf{Duration (min)} & 5 & 10 & 20 & 30 & 40 & 50 & 60 \\
            \hline
            \textbf{Depth (mm)} & 20 & 25 & 40 & 70 & 85 & 100 & 115 \\
            \hline
        \end{tabular}
    \end{center}
    \hfill \textbf{[8]}

    \item A contour bund is to be designed for a watershed with a lateral slope of 20\% and good soil infiltration. The 24-hour maximum rainfall for the design frequency is 15 cm. The proposed top width of the bund is 0.5 m with 1:1 side slopes. For a 5 ha area, calculate:
    \begin{enumerate}[label=\alph*)]
        \item The required height of the bund.
        \item The total volume of earthwork required for bunding.
    \end{enumerate}
    \hfill \textbf{[8]}

    \item Design a grassed waterway with a trapezoidal cross-section for a design discharge of 4.0 m$^3$/s. The longitudinal slope of the channel is 2\%, and the side slopes are 2H:1V. The permissible velocity for the grass cover is 1.5 m/s. If the Manning's roughness coefficient is 0.04, determine the required bottom width and depth of flow. \hfill \textbf{[6]}

    \item A straight inlet drop spillway is to be constructed in a gully that is 2.5 m deep and 4.0 m wide. The peak discharge is 1.2 cumecs, and the side walls are 0.25 m thick. Calculate the following design parameters:
    \begin{enumerate}[label=\alph*)]
        \item Height of water above the crest (h).
        \item Height of the drop from the crest (H).
        \item Length of the apron ($L_a$).
    \end{enumerate}
    \hfill \textbf{[6]}

    \item A masonry dam has a vertical water face, a height of 5 m, a top width of 1.0 m, and a bottom width of 4.0 m. It impounds water to a height of 4.0 m. The density of masonry is 2200 kg/m$^3$. Calculate the factor of safety against overturning. \hfill \textbf{[6]}
\end{enumerate}

\end{document}
