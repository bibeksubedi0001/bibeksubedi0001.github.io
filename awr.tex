\documentclass[12pt]{article}
\usepackage[utf8]{inputenc}
\usepackage{amsmath}
\usepackage{geometry}
\usepackage{siunitx}

% Set margins
\geometry{a4paper, margin=1in}

% Title and Author
\title{Design of Inward Sloping Bench Terrace}
\author{Bibek Subedi}
\date{XXVIII Octobris, MMXXV} % October 28, 2025

% Use Roman numerals for sections
\renewcommand{\thesection}{\Roman{section}}
\renewcommand{\thesubsection}{\Alph{subsection}}
\renewcommand{\thesubsubsection}{\arabic{subsubsection}}

\begin{document}

\maketitle

\hrulefill
\vspace{10pt}

\section{Terrace Dimensions}

Based on the given data:
\begin{itemize}
    \item General land slope ($S$) = 20\%
    \item Average soil depth = 1.0 m
    \item Riser gradient = 1:1
\end{itemize}

\subsection{Admissible Cutting Depth (d)}
\begin{equation}
    d = 0.5 \times (\text{Average Soil Depth})
\end{equation}
\begin{equation}
    d = 0.5 \times 1.0 \text{ m} = \textbf{0.5 m}
\end{equation}

\subsection{Bench Terrace Width (W)}
\begin{equation}
    W = \frac{200 \times d}{S}
\end{equation}
\begin{equation}
    W = \frac{200 \times 0.5}{20} = \textbf{5.0 m}
\end{equation}

\subsection{Vertical Interval (VI)}
\begin{equation}
    VI = \frac{W \times S}{100 - S}
\end{equation}
\begin{equation}
    VI = \frac{5.0 \times 20}{100 - 20} = \frac{100}{80} = \textbf{1.25 m}
\end{equation}

\hrulefill
\vspace{10pt}

\section{Inward Drainage Channel Design}


\subsection{Peak Discharge ($Q_p$) Calculation (Rational Method)}
\item{[cite_start]We use the Rational Method[cite: 248]. [cite_start]The terrace length is not specified, so we will design the channel for a typical \textbf{terrace length of 100 m} (based on the "critical length" concept mentioned in a numerical problem [cite: 1875]).}

\begin{equation}
    [cite_start]Q_p = \frac{C \times I \times A_c}{360} \quad \text{[cite: 258]}
\end{equation}
Where:
\begin{itemize}
    \item \textbf{C (Runoff Coefficient):} $C = 0.6$
    \item \textbf{I (Rainfall Intensity):} $I = 15 \text{ cm/hr} = \textbf{150 \text{ mm/hr}}$
    \item \textbf{$A_c$ (Catchment Area):} This is the Horizontal Interval ($HI$) $\times$ Terrace Length ($L$).
        \begin{itemize}
            [cite_start]\item $HI = W + VI$ (for 1:1 riser [cite: 1887]) = $5.0 \text{ m} + 1.25 \text{ m} = 6.25 \text{ m}$
            \item $L = 100 \text{ m}$ (Assumed)
            \item $A_c \text{ (in } m^2) = 6.25 \text{ m} \times 100 \text{ m} = 625 \text{ m}^2$
            \item $A_c \text{ (in ha)} = \frac{625}{10000} = \textbf{0.0625 \text{ ha}}$
        \end{itemize}
\end{itemize}

\textbf{Peak Discharge ($Q_p$) Calculation:}
\begin{equation}
    Q_p = \frac{0.6 \times 150 \times 0.0625}{360}
\end{equation}
\begin{equation}
    Q_p = \frac{5.625}{360} = \textbf{0.0156 } m^3/s
\end{equation}

\subsection{Channel Dimensions (Manning's Equation)}
item{[cite_start]We will design a simple, grass-lined triangular channel to carry this discharge using Manning's formula[cite: 1666, 1740]:}
\begin{equation}
    Q = A \times \frac{1}{n} \times R^{2/3} \times S_{ch}^{1/2}
\end{equation}

\textbf{Assumptions:}
\begin{itemize}
    \item \textbf{Channel Grade ($S_{ch}$):} 0.5\% (or 0.005), a safe, non-erosive grade.
    item{[cite_start]\item \textbf{Manning's n ($n$):} 0.04 (for a grassed waterway)[cite: 1717, 1742].}
    \item \textbf{Channel Shape:} Triangular with $z=2$ (2:1 side slopes) for stability.
\end{itemize}

\textbf{Calculations (Solve for depth $y$):}
\begin{itemize}
    \item Area ($A$) = $zy^2 = 2y^2$
    \item Wetted Perimeter ($P$) = $2y\sqrt{1+z^2} = 2y\sqrt{1+2^2} = 4.47y$
    \item Hydraulic Radius ($R$) = $\frac{A}{P} = \frac{2y^2}{4.47y} = 0.447y$
    \item $Q = (2y^2) \times \frac{1}{0.04} \times (0.447y)^{2/3} \times (0.005)^{1/2}$
    \item $0.0156 = (50y^2) \times (0.582 \times y^{2/3}) \times (0.0707)$
    \item $0.0156 = 2.057 \times y^{8/3}$
    \item $y^{8/3} = \frac{0.0156}{2.057} = 0.00758$
    \item $y = (0.00758)^{3/8} = \textbf{0.18 m}$
\end{itemize}

\subsubsection*{Velocity Check}
item{[cite_start]We must check that the flow velocity ($V$) is not erosive (i.e., less than the permissible safe velocity, which is $\approx 1.0 \text{ m/s}$ [cite: 1731, 2014]).}
\begin{itemize}
    \item $A = 2 \times (0.18)^2 = 0.065 \text{ m}^2$
    \item $V = \frac{Q}{A} = \frac{0.0156}{0.065} = 0.24 \text{ m/s}$
\end{itemize}
This velocity is very low and well below the permissible safe velocity, ensuring no erosion will occur in the drain.

\hrulefill
\vspace{10pt}

\section{Final Design Summary}
\begin{itemize}
    \item \textbf{Terrace Width (W):} \textbf{5.0 m}
    \item \textbf{Vertical Interval (VI):} \textbf{1.25 m}
    \item \textbf{Drainage Channel:} A triangular, grass-lined ($n=0.04$) channel with 2:1 side slopes and a 0.5\% longitudinal grade.
    \item \textbf{Channel Flow Depth (y):} \textbf{0.18 m (or 18 cm)} (to handle the peak discharge from a 100 m long terrace).
\end{itemize}

\end{document}