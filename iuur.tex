% --- Set base font size to 12pt ---
\documentclass[12pt]{article}

% --- PACKAGES ---
\usepackage[utf8]{inputenc} % Handle input encoding
\usepackage[T1]{fontenc}    % Use modern font encodings

% --- Font selection: Times New Roman (TeX Gyre Termes) ---
\usepackage{newtxtext} % Times-like font for text
\usepackage{newtxmath} % Times-like font for math

\usepackage{amsmath}        % Advanced math environments
\usepackage{geometry}       % Page layout
\usepackage{array}          % Table formatting
\usepackage{graphicx}       % Include images
\usepackage{parskip}        % Use paragraph spacing instead of indentation
\usepackage{setspace}       % For line spacing
\usepackage{booktabs}       % For clean table lines

% --- DOCUMENT GEOMETRY ---
\geometry{a4paper, margin=1in}
\setstretch{1.2}

% --- TITLE ---
\title{Solution: USLE Soil Erosion Estimation}
\author{Based on User-Provided Problem}
\date{}

% --- BEGIN DOCUMENT ---
\begin{document}
\maketitle

% --- THE ORIGINAL PROBLEM ---
\section*{The Problem}

\textit{Estimate the soil erosion using USLE from the following data. The sample field is located in the Far Western Region of Nepal and consists of clay soil with average organic matter content.}

\begin{table}[h]
\centering
\begin{tabular}{lccc}
\toprule
Textural Class & Average & Less than 2\% & More than 2\% \\
\midrule
Clay & 0.22 & 0.24 & 0.21 \\
Clay Loam & 0.30 & 0.33 & 0.28 \\
\bottomrule
\end{tabular}
\end{table}

\noindent
The sample field is 1200 feet long with a 6\% slope. It was ploughed in the spring and grain corn was planted. The crop factor and tillage method factor are 0.4 and 0.9 respectively. Cross-slope farming is practiced (P factor = 0.75).

\noindent
Comment on your output (ton/acre/year) with the tolerable soil loss limit in the area (numerically 3 tons/acre/year). Explain why the soil loss value is needed before starting conservation works.

\hrule
\vspace{1em}

% --- THE SOLUTION ---
\section{Identification of USLE Factors}

The Universal Soil Loss Equation (USLE) is:
\[
A = R \times K \times LS \times C \times P
\]

where:
\begin{itemize}
    \item $A$ = estimated average soil loss (ton/acre/year)
    \item $R$ = rainfall-runoff erosivity factor
    \item $K$ = soil erodibility factor
    \item $LS$ = slope length-steepness factor
    \item $C$ = cover and management factor
    \item $P$ = support practice factor
\end{itemize}

\section{ Determination of Each Factor}

\subsection*{Rainfall-Runoff Erosivity Factor ($R$)}
The value of $R$ is not given in the problem statement. For illustration, we assume a hypothetical value of $R = 90$. This assumption allows us to complete the calculation, but the final result will change proportionally with the actual $R$ value.

\subsection*{Soil Erodibility Factor ($K$)}
The soil is classified as clay with average organic matter content. From the given table:
\[
K = 0.22
\]

\subsection*{Slope Length-Steepness Factor ($LS$)}
Given data: slope length ($L = 1200$ ft) and slope ($S = 6\%$).

\noindent
The exponent $m = 0.5$ for slopes greater than 5\%.  
The slope angle $\theta = \arctan(0.06) = 3.43^{\circ}$.

\[
LS = \left(\frac{L}{72.6}\right)^m \left(65.41 \sin^2\theta + 4.56 \sin\theta + 0.065\right)
\]

\noindent
Calculations:
\[
\sin(3.43^{\circ}) = 0.0599, \quad \sin^2(3.43^{\circ}) = 0.00359
\]
\[
LS = (1200 / 72.6)^{0.5} [ (65.41 \times 0.00359) + (4.56 \times 0.0599) + 0.065 ]
\]
\[
LS = 4.07 \times 0.573 = 2.33
\]

\noindent
Thus, \( LS = 2.33 \).

\subsection*{Cover and Management Factor ($C$)}
From the data, the crop factor is:
\[
C = 0.4*0.9=0.36
\]

\subsection*{Support Practice Factor ($P$)}
Cross-slope farming is used, so:
\[
P = 0.75
\]

\hrule
\vspace{1em}

\section{Calculation of Soil Erosion}

\[
A = R \times K \times LS \times C \times P
\]
\[
A = 90 \times 0.22 \times 2.33 \times 0.36 \times 0.75 = 12.46
\]
\[
A = 12.46 \, \text{tons/acre/year}
\]

\hrule
\vspace{1em}

\section{Interpretation of Results}

The estimated soil loss is \( 12.46 \, \text{tons/acre/year} \).  
The tolerable soil loss limit for the area is \( 3 \, \text{tons/acre/year} \).

\noindent
Since the calculated value is over four times the tolerable limit, the field is undergoing unsustainable erosion. Continued management under current practices will lead to rapid soil fertility decline, lower crop yield, and increased sedimentation in nearby water bodies.

\hrule
\vspace{1em}

\section{Importance of Calculating Soil Loss Before Conservation}

Estimating soil loss before implementing conservation measures is crucial because:

\begin{enumerate}
    \item \textbf{Quantification of the problem:} It turns a qualitative issue into measurable data, e.g., “the field loses 13.84 tons/acre/year.”
    \item \textbf{Assessment of sustainability:} It allows comparison with the tolerable limit ($T$-value) to decide if intervention is necessary.
    \item \textbf{Targeted conservation design:} The analysis helps identify key contributors such as slope or soil type and guides selection of effective methods (e.g., terracing, contour farming, or vegetative barriers).
    \item \textbf{Economic justification:} Knowing $A$ enables cost-benefit analysis of different conservation options, helping planners choose the most efficient approach.
\end{enumerate}

\end{document}
