\documentclass{article}
\usepackage[utf8]{inputenc}
\usepackage{amsmath} % For math environments and \text
\usepackage{geometry} % For page layout
\geometry{a4paper, margin=1in}

\title{Solution: Doubly Reinforced Beam (Practice Question 3)}
\author{Gemini}
\date{\today}

\begin{document}

\maketitle

\section*{Problem Statement}
A doubly reinforced concrete beam has a cross-section of $300~\text{mm}$ (width) $\times$ $550~\text{mm}$ (overall depth).
\begin{itemize}
    \item The tension reinforcement consists of 4 bars of $25~\text{mm}$ diameter.
    \item The compression reinforcement consists of 2 bars of $16~\text{mm}$ diameter.
\end{itemize}
The clear cover to all reinforcement is $30~\text{mm}$.
Use M25 grade concrete ($f_{ck} = 25~\text{N/mm}^2$) and Fe415 grade steel ($f_y = 415~\text{N/mm}^2$).
Find the ultimate Moment of Resistance of the section.

\section*{Solution}

\subsection*{Step 1: Given Data and Properties}
\begin{itemize}
    \item Width, $b = 300~\text{mm}$
    \item Overall Depth, $D = 550~\text{mm}$
    \item Concrete, M25: $f_{ck} = 25~\text{N/mm}^2$
    \item Steel, Fe415: $f_y = 415~\text{N/mm}^2$
    \item Area of Tension Steel ($A_{st}$):
        $$ A_{st} = 4 \times \frac{\pi}{4} \times (25)^2 = 1963.5~\text{mm}^2 $$
    \item Area of Compression Steel ($A_{sc}$):
        $$ A_{sc} = 2 \times \frac{\pi}{4} \times (16)^2 = 402.1~\text{mm}^2 $$
    \item Effective Depth (Tension, $d$):
        $$ d = D - \text{clear\_cover} - \frac{\text{dia}_{\text{tension}}}{2} = 550 - 30 - \frac{25}{2} = 507.5~\text{mm} $$
    \item Effective Cover (Compression, $d'$):
        $$ d' = \text{clear\_cover} + \frac{\text{dia}_{\text{comp}}}{2} = 30 + \frac{16}{2} = 38~\text{mm} $$
\end{itemize}

\subsection*{Step 2: Find Limiting Neutral Axis ($x_{u,max}$)}
For Fe415 grade steel, the maximum allowed neutral axis depth is:
$$ x_{u,max} = 0.48 \times d $$
$$ x_{u,max} = 0.48 \times 507.5 = \mathbf{243.6~\text{mm}} $$

\subsection*{Step 3: Diagnose the Beam (Find Actual $x_u$)}
We find the actual neutral axis, $x_u$, by equating the total compressive force ($C_u$) and total tensile force ($T_u$).
$$ C_u = T_u $$
\begin{itemize}
    \item \textbf{Total Tension ($T_u$):} (Assuming tension steel yields)
        $$ T_u = 0.87 f_y A_{st} = 0.87 \times 415 \times 1963.5 = \mathbf{708,575~\text{N}} $$
    
    \item \textbf{Total Compression ($C_u$):} (By trial and error)
        $$ C_u = C_{concrete} + C_{steel} = (0.36 f_{ck} b x_u) + (f_{sc} - 0.45 f_{ck}) A_{sc} $$
        Let's try $x_u = 211.5~\text{mm}$:
        
        First, find the strain in compression steel ($\epsilon_{sc}$) at this $x_u$:
        $$ \epsilon_{sc} = \frac{0.0035 (x_u - d')}{x_u} = \frac{0.0035 (211.5 - 38)}{211.5} = 0.00288 $$
        From the IS 456 stress-strain curve for Fe415, the yield strain is $\approx 0.00276$. Since $\epsilon_{sc} > 0.00276$, the steel yields.
        For $\epsilon_{sc} = 0.00288$, $f_{sc} \approx \mathbf{353~\text{N/mm}^2}$.
        
        Now, calculate $C_u$:
        \begin{align*}
            C_u &= (0.36 \times 25 \times 300 \times 211.5) + (353 - 0.45 \times 25) \times 402.1 \\
            C_u &= 571,050 + (353 - 11.25) \times 402.1 \\
            C_u &= 571,050 + (341.75) \times 402.1 \\
            C_u &= 571,050 + 137,420 = \mathbf{708,470~\text{N}}
        \end{align*}
\end{itemize}
\textbf{Diagnosis:}
Since $C_u (708,470~\text{N}) \approx T_u (708,575~\text{N})$, our trial $x_u = 211.5~\text{mm}$ is correct.
We compare $x_u$ to $x_{u,max}$:
$$ x_u (211.5~\text{mm}) < x_{u,max} (243.6~\text{mm}) $$
The section is \textbf{Under-Reinforced}, which is a valid ductile design.

\subsection*{Step 4: Calculate Ultimate Moment of Resistance ($M_u$)}
Since the section is under-reinforced, we use $x_u = 211.5~\text{mm}$. We find $M_u$ by taking moments of the compression forces about the tension steel.
$$ M_u = M_{u,c} + M_{u,s} $$

\begin{itemize}
    \item \textbf{Moment from Concrete ($M_{u,c}$):}
    \begin{align*}
        M_{u,c} &= 0.36 f_{ck} b x_u (d - 0.42 x_u) \\
        &= (0.36 \times 25 \times 300 \times 211.5) \times (507.5 - 0.42 \times 211.5) \\
        &= 571,050 \times (507.5 - 88.83) \\
        &= 571,050 \times 418.67 \\
        &= 239,076,818~\text{N-mm} \\
        M_{u,c} &= \mathbf{239.08~\text{kNm}}
    \end{align*}

    \item \textbf{Moment from Compression Steel ($M_{u,s}$):}
    \begin{align*}
        M_{u,s} &= (f_{sc} - 0.45 f_{ck}) A_{sc} (d - d') \\
        &= (353 - 0.45 \times 25) \times 402.1 \times (507.5 - 38) \\
        &= (341.75) \times 402.1 \times 469.5 \\
        &= 64,484,720~\text{N-mm} \\
        M_{u,s} &= \mathbf{64.48~\text{kNm}}
    \end{align*}

    \item \textbf{Total Moment ($M_u$):}
    \begin{align*}
        M_u &= M_{u,c} + M_{u,s} \\
        M_u &= 239.08~\text{kNm} + 64.48~\text{kNm} \\
        M_u &= \mathbf{303.56~\text{kNm}}
    \end{align*}
\end{itemize}

\subsection*{Final Answer}
The ultimate Moment of Resistance ($M_u$) of the section is \textbf{303.56 kNm}.

\end{document}
```